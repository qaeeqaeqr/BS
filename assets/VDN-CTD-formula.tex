\documentclass[UTF8]{ctexart}

\usepackage{amsmath}
\usepackage{amssymb}
\usepackage{amsfonts}

\pagestyle{plain}

\title{多智能体强化学习最小均值方差公式推导}

\begin{document}

\maketitle
\author{王钰}


\section{VDN+CTD 公式推导}

注:这里只考虑VDN和CTD中的Vanilla情况。

\subsection{MDP}

一个MDP可用一个五元组$(S, A, P^a_{s, s^{\prime}}, R, \gamma)$表示。
其中$S$(state)是状态空间,$A$(action)是动作空间,
$P^a_{s, s^{\prime}}$是在状态$s$下采取动作$a$可以转换到状态$s^{\prime}$的概率,
$R_t$是智能体在$t$时可以获得的奖励,在环境中,智能体每次实际获得的奖励记为$r_t$,
$\gamma$是折扣因子。

定义策略为:

$$\pi:P(a|s)$$

策略为从状态$s$选择动作$a$的概率。

定义回报(随机变量)为:

$$G^{\pi}(s_t) = \sum_{k=1}^{\infty} \gamma^{k-1} R_{t+k}$$,

回报为遵从策略$\pi$,智能体从状态$s_t$开始到结束,可以获得的累计折扣奖励。

在MDP中,算法的目标是得到最优的策略$\pi^*$,使得回报$G^{\pi}(s_t)$最大。

\subsection{Value Functions}

定义状态价值函数$V^{\pi}(s_t) = \mathbb{E}[G^{\pi}(s_t)]$,
其中$G^{\pi}(s_t)$是指遵从策略$\pi$,智能体从状态$s_t$开始到结束,可以获得的累计折扣奖励。
对状态价值函数可进行进一步推导,得到状态价值函数的贝尔曼方程:

\begin{align*}
	V^{\pi}(s_t) &= \mathbb{E}[G^{\pi}(s_t)]  \\
	&= \mathbb{E}[R_{t+1} + \gamma R_{t+2} + \gamma^2 R_{t+3} + \cdots]  \\
	&= \mathbb{E}[R_{t+1} + \gamma (R_{t+2} + \gamma R_{t+3} + \cdots)]  \\
	&= \mathbb{E}[R_{t+1} + \gamma G^{\pi}(s_{t+1})] \\
	&= \mathbb{E}[R_{t+1}] + \gamma \mathbb{E}[G^{\pi}(s_{t+1})] \\
\end{align*}


定义状态-动作价值函数$Q^{\pi}(s_t, a_t) = \mathbb{E}[G^{\pi}(s_t, a_t)]$,
其中$G^{\pi}(s_t, a_t)$是指智能体从状态$s_t$开始,选取动作$a_t$,然后遵从策略$\pi$直到结束,可以获得的累计折扣奖励。
同理,对状态-动作价值函数可进行进一步推导,得到状态-动作价值函数的贝尔曼方程:

\begin{align*}
	Q^{\pi}(s_t, a_t) &= \mathbb{E}[G^{\pi}(s_t, a_t)]  \\
	&= \mathbb{E}[R_{t+1} + \gamma R_{t+2} + \gamma^2 R_{t+3} + \cdots]  \\
	&= \mathbb{E}[R_{t+1} + \gamma (R_{t+2} + \gamma R_{t+3} + \cdots)]  \\
	&= \mathbb{E}[R_{t+1} + \gamma G^{\pi}(s_{t+1}, a_{t+1})] \\
	&= \mathbb{E}[R_{t+1}] + \gamma \mathbb{E}[G^{\pi}(s_{t+1}, a_{t+1})] \\
\end{align*}

由于价值函数是回报的期望,所以在最优的策略$\pi^*$时,有最优状态价值函数$V^{\pi^*}(s_t)$
和最优动作-状态价值函数$Q^{\pi^*}(s_t, a_t)$。因此,
找到使得回报$G^{\pi}(s_t)$最大的最优策略$\pi^*$的过程就相当于对价值函数进行优化的过程。

\subsection{TD learning}

强化学习中,一种更新价值函数的方法是TD学习。

根据状态价值函数的贝尔曼方程,可得:

$$V^{\pi}(s_t) = r_{t+1} + \gamma V^{\pi}(s_{t+1})$$

其中$r_{t+1} + \gamma V^{\pi}(s_{t+1})$被称为TD目标,
是即时奖励和下一个状态的价值的无偏估计,用于近似贝尔曼方程中的期望值。

根据TD目标,可定义TD误差$\delta_{vt}$为$r_{t+1} + \gamma V^{\pi}(s_{t+1}) - V^{\pi}(s_t)$。
TD误差项表示实际获得的回报与预期回报之间的差异。

利用TD误差,可以推导出更新当前的状态价值函数的公式:

\begin{align*}
	V^{\pi}(s_t) &= V^{\pi}(s_t) + \alpha \times \delta_{vt}  \\
	&= V^{\pi}(s_t) + \alpha (r_{t+1} + \gamma V^{\pi}(s_{t+1}) - V^{\pi}(s_t))
\end{align*}

其中$\alpha$是学习率,控制状态价值函数的更新速度。

同理,根据状态-动作价值函数的贝尔曼方程,可得:

$$Q^{\pi}(s_t, a_t) = r_{t+1} + \gamma Q^{\pi}(s_{t+1}, a_{t+1})$$

同理,可以推导出更新当前状态-动作价值函数的公式:

$$
Q^{\pi}(s_t, a_t) = Q^{\pi}(s_t, a_t) + \alpha (r_{t+1} + \gamma Q^{\pi}(s_{t+1}, a_{t+1}) - Q^{\pi}(s_t, a_t))
$$

其中$\alpha$是学习率,控制状态-动作价值函数的更新速度;可定义TD误差$\delta_{qt}$为$r_{t+1} + \gamma Q^{\pi}(s_{t+1}, a_{t+1}) - Q^{\pi}(s_t, a_t)$。

\subsection{CTD}

\subsubsection{用于计算方差的价值函数推导}

根据定义$V^{\pi}(s_t) = \mathbb{E}[G^{\pi}(s_t)]$,前面的推导可以获得计算在状态$s_t$下的回报的均值。接下来,要推导出可以计算在状态$s_t$下的回报的方差$\bar{V}^{\pi}(s_t)$。

首先,对回报的方差$\mathbb{V}[G^{\pi}(s_t)]$进行推导:

\begin{align*}
	\mathbb{V}[G^{\pi}(s_t)] &= \mathbb{E}[ G^{\pi}(s_{t}) - \mathbb{E}[G^{\pi}(s_t)]]^2 \\
	&= \mathbb{E}[r_{t+1} + G^{\pi}(s_{t+1}) - \mathbb{E}[G^{\pi}(s_t)]]^2 \\
	&= \mathbb{E}[r_{t+1} + G^{\pi}(s_{t+1}) - V^{\pi}(s_t)]^2 \\
	&= \mathbb{E}[r_{t+1} + G^{\pi}(s_{t+1}) - V^{\pi}(s_t) + V^{\pi}(s_{t+1}) - V^{\pi}(s_{t+1})]^2 \\
	&= \mathbb{E}[(r_{t+1} + V^{\pi}(s_{t+1}) - V^{\pi}(s_t)) + (G^{\pi}(s_{t+1}) - V^{\pi}(s_{t+1}))]^2 \\
	&= \mathbb{E}[r_{t+1} + V^{\pi}(s_{t+1}) - V^{\pi}(s_t)]^2 + \mathbb{E}[G^{\pi}(s_{t+1}) - V^{\pi}(s_{t+1})]^2 \\
	&\qquad+ 2 \times \mathbb{E}[r_{t+1} + V^{\pi}(s_{t+1}) - V^{\pi}(s_t)] \times  \mathbb{E}[G^{\pi}(s_{t+1}) - V^{\pi}(s_{t+1})] \\
	&= \mathbb{E}[r_{t+1} + V^{\pi}(s_{t+1}) - V^{\pi}(s_t)]^2 + \mathbb{V}[G^{\pi}(s_{t+1})] \\
	&\qquad + 2 \times \mathbb{E}[r_{t+1} + V^{\pi}(s_{t+1}) - V^{\pi}(s_t)] \times 0 \\
	&= \mathbb{E}[\delta_{vt}]^2 + \mathbb{V}[G^{\pi}(s_{t+1})] \\
\end{align*}

其中,$\mathbb{V}$表示计算某个变量的方差,根据先前的定义,$\mathbb{E}[G^{\pi}(s_t)] = V^{\pi}(s_t)$...(论文里这里每行推导都可以细讲)。

回顾TD学习,用于计算回报的均值的状态价值函数的贝尔曼方程为:$V^{\pi}(s_t) = r_{t+1} + \gamma V^{\pi}(s_{t+1})$;用于计算回报的均值的状态价值函数的更新公式为:$V^{\pi}(s_t) = V^{\pi}(s_t) + \alpha (r_{t+1} + \gamma V^{\pi}(s_{t+1}) - V^{\pi}(s_t))$。因此,根据以上关于$\mathbb{V}[G^{\pi}(s_t)]$的推导,可得出用于计算回报的方差的状态价值函数的更新公式为:

$$
\bar{V}^{\pi}(s_t) = \bar{V}^{\pi}(s_t) + \bar{\alpha}(\delta_{vt}^2 + \bar{V}^{\pi}(s_{t+1}) - \bar{V}^{\pi}(s_t))
$$

在这里,我们采用了TD学习中的常用技术,用智能体与环境交互中实际计算的$\delta_{vt}^2$来代替$\mathbb{E}[\delta_{vt}]^2$。类似的,可以得到用于计算回报的方差的状态-动作价值函数的更新公式为:

$$
\bar{Q}^{\pi}(s_t, a_t) = \bar{Q}^{\pi}(s_t, a_t) + \bar{\alpha}(\delta_{qt}^2 + \bar{Q}^{\pi}(s_{t+1}, a_{t+1}) - \bar{Q}^{\pi}(s_t, a_t))
$$


\subsection{DQN}




\subsection{VDN}

	
	
\end{document}